\documentclass{article}

\usepackage[margin=1in]{geometry} 
\usepackage{amsmath, amsthm, amssymb, hyperref}

\newcommand{\R}{\mathbb{R}}  
\newcommand{\Z}{\mathbb{Z}}
\newcommand{\N}{\mathbb{N}}
\newcommand{\Q}{\mathbb{Q}}
\newcommand{\F}{\mathbb{F}}
\newcommand{\C}{\mathbb{C}}
\newcommand{\D}{\mathbb{D}}
\newcommand{\bH}{\mathbb{H}}

\newcommand{\abs}[1]{\left| #1 \right|}
\newcommand{\set}[1]{\left\{ #1 \right\}}
\newcommand{\brac}[1]{\left[ #1 \right]}
\newcommand{\paren}[1]{\left( #1 \right)}

\DeclareMathOperator{\real}{Re}
\DeclareMathOperator{\imag}{Im}

\newcommand{\comp}[2]{#1 \circ #2}

\DeclareMathOperator{\dom}{dom}

\newenvironment{theorem}[2][Theorem]{\begin{trivlist}
\item[\hskip \labelsep {\bfseries #1}\hskip \labelsep {\bfseries #2.}]}{\end{trivlist}}
\newenvironment{lemma}[2][Lemma]{\begin{trivlist}
\item[\hskip \labelsep {\bfseries #1}\hskip \labelsep {\bfseries #2.}]}{\end{trivlist}}
\newenvironment{exercise}[2][Exercise]{\begin{trivlist}
\item[\hskip \labelsep {\bfseries #1}\hskip \labelsep {\bfseries #2.}]}{\end{trivlist}}
\newenvironment{problem}[2][Problem]{\begin{trivlist}
\item[\hskip \labelsep {\bfseries #1}\hskip \labelsep {\bfseries #2.}]}{\end{trivlist}}
\newenvironment{question}[2][Question]{\begin{trivlist}
\item[\hskip \labelsep {\bfseries #1}\hskip \labelsep {\bfseries #2.}]}{\end{trivlist}}
\newenvironment{corollary}[2][Corollary]{\begin{trivlist}
\item[\hskip \labelsep {\bfseries #1}\hskip \labelsep {\bfseries #2.}]}{\end{trivlist}}

\newenvironment{solution}{\begin{proof}[Solution]}{\end{proof}}
\newenvironment{definition}{\begin{proof}[Definition]}{\end{proof}}

% lol, lmao
\linespread{1.1}
\renewcommand{\baselinestretch}{1.17}

% from 117
    % \newenvironment{practice}
    % [2]{\begin{trivlist}
    % \item[\hskip \labelsep {\bfseries     
    % % "debug mode": for working on it
    % % Practice #1, Week #2.
    % % "release mode": for submitting
    % Week #2: Practice #1.
    % }\hskip \labelsep]}{\end{trivlist}}
    
\begin{document}

% 117
% \renewcommand{\labelenumi}{(\alph{enumi})}

% ------------------------------------------ %
%                 START HERE                 %
% ------------------------------------------ %

\title{Homework 1} % Replace with appropriate title
\author{Nathaniel Hamovitz\\Math 118A, Garcia-Cervera, F22}
\date{due 2022-10-02}

\maketitle

\begin{problem}{1} % ✅
    Prove that $\sqrt{2} + \sqrt{3}$ is irrational.
\end{problem}
\begin{proof}
    We use the rational zeroes theorem. Let $x = \sqrt{2} + \sqrt{3}$. Then we have:
    \begin{align*}
        x &= \sqrt{2} + \sqrt{3} \\
        x^2 &= 2 + 2\sqrt{6} + 3 \\
        x^2 - 5 &= \sqrt{6} \\
        x^4 - 10x^2 + 25 &= 6 \\
        x^4 - 10x^2 + 19 &= 0 \\
    \end{align*}


    By the rational zeroes theorem, any rational solution of this equation must be an integer that divides $19$. The only such integers are $\pm 1$ and $\pm 19$, and when those candidates are subsituted back in for $x$, we get $10$ and $126730$. Therefore $x = \sqrt{2} + \sqrt{3}$ is irrational.
\end{proof}

% ---

\begin{problem}{2}
    Given nonempty subsets $A$ and $B$ of $\R$, let $C$ denote the set
    $$C = \{x + y : x \in A, y \in B\}.$$
    If each $A$ and $B$ has a supremum, then prove that $C$ has a supremum and
    $$\sup C = \sup A + \sup B.$$    
\end{problem}
\begin{proof}
    It suffices to show that $\sup C \le \sup A + \sup B$ and $\sup C \ge \sup A + \sup B$.

    For all $a \in A$ and $b \in B$, $a + b \le \sup A + b \le \sup A + \sup B$. Therefore $\sup A + \sup B$ is an upper bound of $C$, and so by the least upper bound property, $\sup C$ exists. Further, by the definition of supremum $\sup C \le \sup A + \sup B$.

    Suppose for a contradiction that there exists an upper bound $x$ of $C$ such that $x < \sup A + \sup B$. Then $x - \sup A < \sup B$, which implies $\exists b \in B, x - \sup A < b < \sup B$, and thus that $x - b < \sup A$. Again, this implies that $\exists a \in A, x - b < a < \sup A$, and thus that $x < a + b$. This is a contradiction, as we assumed that $x$ was an upper bound of $C$. Since $\sup C$ is an upper bound of $C$, we have $\sup C \ge \sup A + \sup B$.

    Therefore $\sup C = \sup A + \sup B$.
\end{proof}

% ---

\begin{problem}{3}
    Given nonempty subsets $S$ and $T$ of $\R$ such that $S \subseteq T$ and $T$ has a supremum.
\end{problem}
\begin{enumerate}
    \item[(a).] Prove that $S$ has a supremum and $\sup S \le \sup T$.
    \begin{proof}
        Since $T$ has a supremum and $S \subseteq T$, $S$ must be bounded above. We can see that $\sup T$ is an upper bound of $s$; $\forall s in S, s \in T$ and $\forall t \in T, t \le \sup T$. As $S$ is bounded above and $S \subseteq \R$, $S$ has a supremum by the least upper bound property.

        By definition, $\forall t \in T, t \le \sup T$. Since $S \subseteq T$, we have also that $\forall s \in S, s \le \sup T$. Therefore $\sup T$ is an upper bound of $S$. However, $\sup S$ is by definition the least upper bound of $S$, and so $\sup S \le \sup T$.
    \end{proof}.



    \item[(b).] Give an example of two sets $S$ and $T$ in the real line such that $S \subset T$ (strictly contained), and $\sup S = \sup T$.
    \begin{solution}
        $S = (0, 1)$ and $T = [0, 1]$.
    \end{solution}
    
    
    \item[(c).] Is is true that if two sets satisfy $\sup S = \sup T$ then necessarily $S \subseteq T$? Prove it or give a counterexample.
    \begin{proof}
        It is not; one counterexample is $S = [-1, 1]$ and $T = [0, 1]$.        
    \end{proof}
\end{enumerate}

% ---

\begin{problem}{4}
    Let $A$ be a nonempty set of real numbers which is bounded below. Let $-A$ be the set of all numbers $-x$, where $x \in A$. Prove that $\inf A = - \sup(-A)$.
\end{problem}

\begin{proof}
    Since $A$ is bounded below, $\inf A$ must exist. By definition of infimum, $\forall a \in A, \inf A \le a$ and thus $-a \le -\inf A$. Hence $-\inf A$ is an upper bound for $-A$, and so $-A$ must have a supremum.

    Now we show that $-\inf A$ is the least upper bound of $-A$. Let $t \in \R$; suppose $t$ is an upper bound of $-A$. Then we have that $\forall a \in A, t \ge -a$ or equivalently that $-t \le a$. Assume for a contradiction that $t < -\inf A$. Then $\inf A < -t$, and so $\exists x \in A$ such that $\inf A < x < -t$. This is a contradiction as if $t$ is an upper bound of $-A$ then $\forall a \in A, a \ge -t$.
    
    Therefore $\sup(-A) = -\inf A$; multiply by $-1$ and we have finally that $\inf A = - \sup (-A)$.


\end{proof}

\end{document}