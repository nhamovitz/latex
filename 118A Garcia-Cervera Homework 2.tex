\documentclass{article}

\usepackage[margin=1in]{geometry} 
\usepackage{amsmath, amsthm, amssymb, hyperref}

\newcommand{\R}{\mathbb{R}}  
\newcommand{\Z}{\mathbb{Z}}
\newcommand{\N}{\mathbb{N}}
\newcommand{\Q}{\mathbb{Q}}
\newcommand{\F}{\mathbb{F}}
\newcommand{\C}{\mathbb{C}}
\newcommand{\D}{\mathbb{D}}
\newcommand{\bH}{\mathbb{H}}

\newcommand{\abs}[1]{\left| #1 \right|}
\newcommand{\set}[1]{\left\{ #1 \right\}}
\newcommand{\brac}[1]{\left[ #1 \right]}
\newcommand{\paren}[1]{\left( #1 \right)}

\DeclareMathOperator{\real}{Re}
\DeclareMathOperator{\imag}{Im}

\newcommand{\comp}[2]{#1 \circ #2}

\DeclareMathOperator{\dom}{dom}

\newenvironment{theorem}[2][Theorem]{\begin{trivlist}
\item[\hskip \labelsep {\bfseries #1}\hskip \labelsep {\bfseries #2.}]}{\end{trivlist}}
\newenvironment{lemma}[2][Lemma]{\begin{trivlist}
\item[\hskip \labelsep {\bfseries #1}\hskip \labelsep {\bfseries #2.}]}{\end{trivlist}}
\newenvironment{exercise}[2][Exercise]{\begin{trivlist}
\item[\hskip \labelsep {\bfseries #1}\hskip \labelsep {\bfseries #2.}]}{\end{trivlist}}
\newenvironment{problem}[2][Problem]{\begin{trivlist}
\item[\hskip \labelsep {\bfseries #1}\hskip \labelsep {\bfseries #2.}]}{\end{trivlist}}
\newenvironment{question}[2][Question]{\begin{trivlist}
\item[\hskip \labelsep {\bfseries #1}\hskip \labelsep {\bfseries #2.}]}{\end{trivlist}}
\newenvironment{corollary}[2][Corollary]{\begin{trivlist}
\item[\hskip \labelsep {\bfseries #1}\hskip \labelsep {\bfseries #2.}]}{\end{trivlist}}

\newenvironment{solution}{\begin{proof}[Solution]}{\end{proof}}
\newenvironment{definition}{\begin{proof}[Definition]}{\end{proof}}

% lol, lmao
\linespread{1.1}
\renewcommand{\baselinestretch}{1.17}

% from 117
    % \newenvironment{practice}
    % [2]{\begin{trivlist}
    % \item[\hskip \labelsep {\bfseries     
    % % "debug mode": for working on it
    % % Practice #1, Week #2.
    % % "release mode": for submitting
    % Week #2: Practice #1.
    % }\hskip \labelsep]}{\end{trivlist}}
    
\begin{document}

% 117
% \renewcommand{\labelenumi}{(\alph{enumi})}

% ------------------------------------------ %
%                 START HERE                 %
% ------------------------------------------ %

\title{Homework 2} % Replace with appropriate title
\author{Nathaniel Hamovitz\\Math 118A, Garcia-Cervera, F22}
\date{due 2022-10-09}

\maketitle

\textbf{1. } % ✅
Let $x \in \R$ and assume that $0 \le x \le \varepsilon$ for all $\varepsilon > 0$. Prove that $x = 0$.

\begin{proof}
    By hypothesis, $0 \le x$. Suppose for a contradiction that $x > 0$. Let $\varepsilon = \frac{x}{2} > 0$. Then $\varepsilon < x$, which is a contradiciton.    
\end{proof}


\newpage % ---


\textbf{2. }
Given nonempty subsets $A$ and $B$ of positive real numbers, let $C$ denote the set
$$C = \{xy : x \in A, y \in B\}.$$

If each $A$ and $B$ has a supremum, then prove that $C$ has a supremum and
$$\sup C = \sup A \cdot \sup B.$$

\begin{proof}
    
\end{proof}


\newpage % ---


\textbf{3. }
Check if the following sets are bounded, and if so, find their suprema and infima. Determine whether the suprema and infima belong to the set:

\renewcommand{\labelenumi}{(\alph{enumi}).}
\begin{enumerate}
    \item
    $\displaystyle A = \set{\frac{x + 1}{\abs{x} + 2} : x \in \R}$
    \begin{solution}
        $A$ is bounded above and below. $\inf A = -1$ and $\sup A = 1$; neither is a member of $A$.
    \end{solution}


    \item 
    $\displaystyle B = \set{\frac{2}{n} - \frac{3}{m} : n, m \in \N}$
    \begin{solution}
        $B$ is bounded above and below. $\inf B = -3$ and $\sup B = 2$; neither is a member of $B$.

    \end{solution}


    \item 
    $\displaystyle C = \set{x \in \R : \abs{\abs{x - 1} - \abs{x - 2}} < 2}$
    \begin{solution}
        $C$ is bounded above and below. $\inf C = 0$ and $\sup C = 1$; both are members of $C$.

    \end{solution}
\end{enumerate}


\newpage % ---


\textbf{4. }
Let $S$ and $T$ be nonempty subsets of $\R$ such that $s \le t$ for all $s \in S$ and $t \in T$. Prove that $S$ has a supremum, $T$ has an infimum, and
$$\sup S \le \inf T.$$

\begin{proof}
    $S$ is bounded above (by every $t \in T$) and $T$ is bounded below (by every $s \in S$). Therefore, by the Axiom of Completeness, $\sup S$ and $\inf T$ must exist.

    Suppose for a contradiction that $\sup S > \inf T$. Let $\varepsilon = \sup S - \inf T > 0$. By the definition of $\sup$, $\exists s \in S, \sup S - \frac{\varepsilon}{2} < s \le \sup S$. By the definition of $\inf$, $\exists t \in T, \inf T \le t < \inf T + \frac{\varepsilon}{2}$.
    % add detail here
    Then $t < s$, which is a contradiction.    
\end{proof}










\end{document}