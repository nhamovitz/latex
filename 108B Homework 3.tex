\documentclass{article}

\usepackage[margin=1in]{geometry} 
\usepackage{amsmath, amsthm, amssymb, hyperref}

\newcommand{\R}{\mathbb{R}}  
\newcommand{\Z}{\mathbb{Z}}
\newcommand{\N}{\mathbb{N}}
\newcommand{\Q}{\mathbb{Q}}
\newcommand{\F}{\mathbb{F}}
\newcommand{\C}{\mathbb{C}}
\newcommand{\D}{\mathbb{D}}
\newcommand{\bH}{\mathbb{H}}

\newcommand{\abs}[1]{\left| #1 \right|}
\newcommand{\set}[1]{\left\{ #1 \right\}}
\newcommand{\brac}[1]{\left[ #1 \right]}
\newcommand{\paren}[1]{\left( #1 \right)}
\newcommand{\norm}[1]{\left\Vert #1 \right\Vert}
\newcommand{\ol}[1]{\overline{#1}}
\newcommand{\iprod}[2]{\left\langle #1, #2 \right\rangle}


\DeclareMathOperator{\real}{Re}
\DeclareMathOperator{\imag}{Im}

\newcommand{\comp}[2]{#1 \circ #2}

\DeclareMathOperator{\dom}{dom}
\DeclareMathOperator{\range}{range}

\newenvironment{theorem}[2][Theorem]{\begin{trivlist}
\item[\hskip \labelsep {\bfseries #1}\hskip \labelsep {\bfseries #2.}]}{\end{trivlist}}
\newenvironment{lemma}[2][Lemma]{\begin{trivlist}
\item[\hskip \labelsep {\bfseries #1}\hskip \labelsep {\bfseries #2.}]}{\end{trivlist}}
\newenvironment{exercise}[2][Exercise]{\begin{trivlist}
\item[\hskip \labelsep {\bfseries #1}\hskip \labelsep {\bfseries #2.}]}{\end{trivlist}}
\newenvironment{problem}[2][Problem]{\begin{trivlist}
\item[\hskip \labelsep {\bfseries #1}\hskip \labelsep {\bfseries #2.}]}{\end{trivlist}}
\newenvironment{question}[2][Question]{\begin{trivlist}
\item[\hskip \labelsep {\bfseries #1}\hskip \labelsep {\bfseries #2.}]}{\end{trivlist}}
\newenvironment{corollary}[2][Corollary]{\begin{trivlist}
\item[\hskip \labelsep {\bfseries #1}\hskip \labelsep {\bfseries #2.}]}{\end{trivlist}}

\newenvironment{solution}{\begin{proof}[Solution]}{\end{proof}}
\newenvironment{definition}{\begin{proof}[Definition]}{\end{proof}}

% lol, lmao
\linespread{1.1}
\renewcommand{\baselinestretch}{1.17}

% from 117
    % \newenvironment{practice}
    % [2]{\begin{trivlist}
    % \item[\hskip \labelsep {\bfseries     
    % % "debug mode": for working on it
    % % Practice #1, Week #2.
    % % "release mode": for submitting
    % Week #2: Practice #1.
    % }\hskip \labelsep]}{\end{trivlist}}
    
\begin{document}

% 117
\renewcommand{\labelenumi}{(\alph{enumi})}

% ------------------------------------------ %
%                 START HERE                 %
% ------------------------------------------ %

\title{Homework 3} % Replace with appropriate title
\author{Nathaniel Hamovitz\\Math 108B, Sung, F22}
\date{due 2022-10-27}

\maketitle

% Axler second edition, chapter 6: 19, 25, 26, 30, 31.

\textbf{Axler 2e, Exercise 6.19: }
Suppose $T \in \mathcal{L}(V)$ and $U$ is a subspace of $V$. Prove that $U$ is invariant under $T$ iff $P_U T P_U = T P_U$.

\begin{proof}
    ($\Longrightarrow$) Suppose that $U$ is invariant under $T$. Then $\forall u \in U, Tu \in U$. Let $v \in V$. Then $v = u + u'$ where $u \in U$ and $u' \in U^\perp$. By the definition of the orthogonal projection, $P_U v = u$. Because $U$ is invariant under $T$, $T u = u$. And because $u \in U$, $P_U$ is the identity. Thus the action of $P_U T P_U$ and $T P_U$ is the same for all $v \in V$. 

    ($\Longleftarrow$) Suppose that $P_U T P_U $
    
\end{proof}


\newpage % ---


\textbf{Axler 2e, Exercise 6.25: }
Find a polynomial $q \in \mathcal{P}_2(\R)$ such that
$$\int_0^1 p(x) \cos(\pi x) \: dx = \int_0^1 p(x) q(x) \: dx$$
for every $p \in \mathcal{P}_2(\R)$.

\begin{proof}
    % 3e Thm 6.42 and Ex 6.44, p 188
    
\end{proof}


\newpage % ---


\textbf{Axler 2e, Exercise 6.26: }
Fix a vector $v \in V$ and define $T \in \mathcal{L}(V, \F)$ by $Tu = \iprod{u}{v}$. For $a \in \F$, find a formula for $T^* a$.

\begin{proof}
    
\end{proof}


\newpage % ---


\textbf{Axler 2e, Exercise 6.30: }
Suppose $T \in \mathcal{L}(V, W)$. Prove that

\begin{enumerate}
    \item 
    $T$ is injective iff $T^*$ is surjective.
    \begin{proof}
        ($\Longrightarrow$) Suppose $T$ is injective. Then $\forall u, v \in V, u \ne v \Rightarrow Tu \ne Tv$.

        Therefore $T^*$ is surjective.
    \end{proof}


    \item 
    $T$ is surjective iff $T*$ is injective.
    \begin{proof}
        
    \end{proof}
\end{enumerate}


\newpage % ---


\textbf{Axler 2e, Exercise 6.31: }
Prove that
$$\dim \null T^* = \dim \null T + \dim W - \dim V$$
and
$$\dim \range T^* = \dim \range T$$
for every $T \in \mathcal{L}(V, W)$.
\begin{proof}
    
\end{proof}


\end{document}