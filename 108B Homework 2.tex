\documentclass{article}

\usepackage[margin=1in]{geometry} 
\usepackage{amsmath, amsthm, amssymb, hyperref}

\newcommand{\R}{\mathbb{R}}  
\newcommand{\Z}{\mathbb{Z}}
\newcommand{\N}{\mathbb{N}}
\newcommand{\Q}{\mathbb{Q}}
\newcommand{\F}{\mathbb{F}}
\newcommand{\C}{\mathbb{C}}
\newcommand{\D}{\mathbb{D}}
\newcommand{\bH}{\mathbb{H}}

\newcommand{\abs}[1]{\left| #1 \right|}
\newcommand{\set}[1]{\left\{ #1 \right\}}
\newcommand{\brac}[1]{\left[ #1 \right]}
\newcommand{\paren}[1]{\left( #1 \right)}
\newcommand{\norm}[1]{\left\Vert #1 \right\Vert}
\newcommand{\iprod}[2]{\left\langle #1, #2 \right\rangle}

\DeclareMathOperator{\real}{Re}
\DeclareMathOperator{\imag}{Im}

\newcommand{\comp}[2]{#1 \circ #2}

\DeclareMathOperator{\dom}{dom}

\newenvironment{theorem}[2][Theorem]{\begin{trivlist}
\item[\hskip \labelsep {\bfseries #1}\hskip \labelsep {\bfseries #2.}]}{\end{trivlist}}
\newenvironment{lemma}[2][Lemma]{\begin{trivlist}
\item[\hskip \labelsep {\bfseries #1}\hskip \labelsep {\bfseries #2.}]}{\end{trivlist}}
\newenvironment{exercise}[2][Exercise]{\begin{trivlist}
\item[\hskip \labelsep {\bfseries #1}\hskip \labelsep {\bfseries #2.}]}{\end{trivlist}}
\newenvironment{problem}[2][Problem]{\begin{trivlist}
\item[\hskip \labelsep {\bfseries #1}\hskip \labelsep {\bfseries #2.}]}{\end{trivlist}}
\newenvironment{question}[2][Question]{\begin{trivlist}
\item[\hskip \labelsep {\bfseries #1}\hskip \labelsep {\bfseries #2.}]}{\end{trivlist}}
\newenvironment{corollary}[2][Corollary]{\begin{trivlist}
\item[\hskip \labelsep {\bfseries #1}\hskip \labelsep {\bfseries #2.}]}{\end{trivlist}}

\newenvironment{solution}{\begin{proof}[Solution]}{\end{proof}}
\newenvironment{definition}{\begin{proof}[Definition]}{\end{proof}}

% lol, lmao
\linespread{1.1}
\renewcommand{\baselinestretch}{1.17}

% from 117
    % \newenvironment{practice}
    % [2]{\begin{trivlist}
    % \item[\hskip \labelsep {\bfseries     
    % % "debug mode": for working on it
    % % Practice #1, Week #2.
    % % "release mode": for submitting
    % Week #2: Practice #1.
    % }\hskip \labelsep]}{\end{trivlist}}
    
\begin{document}

% 117
% \renewcommand{\labelenumi}{(\alph{enumi})}

% ------------------------------------------ %
%                 START HERE                 %
% ------------------------------------------ %

\title{Homework 2} % Replace with appropriate title
\author{Nathaniel Hamovitz\\Math 108B, Sung, F22}
\date{due 2022-10-20}

\maketitle

% Chapter 5: 18, 19
% Chapter 6: , 1, 8, 10, 11, 14, 15, 18


\textbf{Axler 2e, Exercise 5.18: } % ✅
Give an example of an operator whose matrix with respect to some basis contains only 0's on the diagonal, but the operator is invertible.

\begin{proof}[Solution]
    Consider
    $$T = \begin{pmatrix}
        0 & 1 \\
        1 & 0
    \end{pmatrix},$$
    which is its own inverse.
\end{proof}


\newpage % ---


\textbf{Axler 2e, Exercise 5.19: } % ✅
Give an example of an operator whose matrix with respect to some basis contains only nonzero numbers on the diagonal, but the operator is not invertible. 

\begin{proof}[Solution]
    Consider $$T = \begin{pmatrix}
        1 & 1 \\
        1 & 1
    \end{pmatrix}.$$

    $T$ is not invertible because it is not injective; note that (for example) $T(2, 0)^t = T(1, 1)^t = (2, 2)^t$.    
\end{proof}


\newpage % ---


\textbf{Axler 2e, Exercise 6.1: }
Prove that if $x, y$ are nonzero vectors in $\R^2$, then
$$\iprod{x}{y} = \norm{x} \norm{y} \cos \theta,$$
where $\theta$ is the angle between $ x$ and $y$ (thinking of $x$ and $y$ as arrows with initial point at the origin).
% Hint: draw the triangle formed by x, y, and (x - y); then use the Law of Cosines

\begin{proof}
    Consider the traingle formed by the vectors $x$, $y$, and $x - y$. From the Law of Cosines (remember high school geometry?) we then have that
    $$\norm{x - y}^2 = \norm{x}^2 + \norm{y}^2 - 2 \norm{x} \norm{y} \cos \theta.$$

    Applying some properties of the norm and the inner product, we see
    \begin{align*}
        \norm{x - y}^2 &= \norm{x}^2 + \norm{y}^2 - 2 \norm{x} \norm{y} \cos \theta \\
        \iprod{x-y}{x-y} &= \\
        \iprod{x}{x} - \iprod{x}{y} - \iprod{y}{x} + \iprod{y}{y} &=  \\
        \norm{x}^2 + \norm{y}^2 - 2\iprod{x}{y} &= \norm{x}^2 + \norm{y}^2 - 2 \norm{x} \norm{y} \cos \theta\\
        - 2\iprod{x}{y} &= - 2 \norm{x} \norm{y} \cos \theta\\
        \iprod{x}{y} &= \norm{x} \norm{y} \cos \theta
    \end{align*}
\end{proof}

\newpage % ---

% \textbf{Axler 2e, Exercise 6.8: }
% A norm on a vector space $U$ is a function $\norm{\quad}: \mathit{U} \mapsto [0, \infty)$ such that $\norm{u} = 0$ iff $u = 0$; $\norm{\alpha u} = \abs{\alpha} \norm{u}$ for all $\alpha \in \F$ and all $u \in \mathit{U}$, and $\norm{u + v} \le \norm{u} + \norm{v}$ for all $u, v \in \mathit{U}$. Prove that a norm satisfying the parallelogram equality comes from the an inner product (in other words, show that if $\norm{\quad}$ is a norm on $U$ satisfying the parellelogram equality, then there is an inner product $\iprod{\cdot}{\cdot}$ on $U$ such that $\norm{u} = \iprod{u}{u}^{\frac{1}{2}}$ for all $u \in U$).

% \begin{proof}
    
% \end{proof}


% \newpage % ---


\textbf{Axler 2e, Exercise 6.10: }
On $\mathcal{P}_2(\R)$, consider the inner product given by
$$\iprod{p}{q} = \int_0^1 p(x) q(x) \: dx.$$
Apply the Gram-Schmidt prodedure to the basis $(1, x, x^2)$ to produce an orthonormal basis of $\mathcal{P}_2(\R)$.
% I think this is a worked example in 3e
% edit: oh, no, it's a hw exercise in 3e as well. the worked example is with the integral from -1 to 1

\begin{proof}[Solution]
    Let $e_1 = \frac{v_1}{\norm{v_1}} = \frac{1}{\sqrt{\iprod{1}{1}}} = \frac{1}{\sqrt{1}} = 1$. 

    Let \begin{align*}
        e_2 &= \frac{v_2 - \iprod{v_2}{e_1}e_1}{\norm{v_2 - \iprod{v_2}{e_1}e_1}} \\
        &= \frac{x - \iprod{x}{1}\cdot 1}{\norm{x - \iprod{x}{1} \cdot 1}} \\
        &= \frac{x - \paren{\int_0^1 x \: dx}(1)}{\norm{x - \paren{\int_0^1 x \: dx}(1)}} \\
        &= \frac{x - \frac{1}{2}}{\norm{x - \frac{1}{2}}} \\
        &= \paren{x - \frac{1}{2}}\paren{\iprod{x - \frac{1}{2}}{x - \frac{1}{2}}^2x}^{-1} \\
        &= \paren{x - \frac{1}{2}}\paren{\int_0^1 \paren{x - \frac{1}{2}}^2}^{-1} \\
        &= 12\paren{x - \frac{1}{2}} \\
        &= 12x - 6
    \end{align*}

    
\end{proof}

\newpage % ---

\textbf{Axler 2e, Exercise 6.11: }
What happens if the Gram-Schmidt procedure is applied to a list of vectors that is not linearly independent?

\begin{proof}[Solution]
    
\end{proof}


\newpage % ---


\textbf{Axler 2e, Exercise 6.14: }
Find an orthonormal basis of $\mathcal{P}_2(\R)$ (with inner product as in Exercise 10) such that the differentiation operator (the operator that takes $p$ to $p'$) on $\mathcal{P}_2(\R)$ has an upper-triangular matrix with respect to this basis.

\begin{proof}[Solution]
    
\end{proof}


\newpage % ---


\textbf{Axler 2e, Exercise 6.15: } % ✅
Suppose $U$ is a subspace of $V$. Prove that
$$\dim U^\perp = \dim V - \dim U.$$

\begin{proof}
    % this is 6.50 in Axler 3e, p 195

    % Note: Briefly consider the case where V is not finite-dim. It's kinda nonsencical 

    This follows from the facts that the direct sum of any subspace and it's orthogonal complement is the whole vector space, and that the dimension of a direct sum is the sum of the dimensions of it's components.    
\end{proof}


\newpage % ---


\textbf{Axler 2e, Exercise 6.18: }
Prove that if $P \in \mathcal{L}(V)$ is such that $P^2 = P$ and
$$\norm{Pv} \le \norm{v}$$
for every $v \in V$, then $P$ is an orthogonal projection.

\begin{proof}
    Let $P \in \mathcal{L}(V)$. Suppose that $P^2 = P$ and $\forall v \in V, \norm{Pv} \le \norm{ v}$.
\end{proof}

\end{document}