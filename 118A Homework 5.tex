\documentclass{article}

\usepackage[margin=1in]{geometry} 
\usepackage{amsmath, amsthm, amssymb, hyperref}

\newcommand{\R}{\mathbb{R}}  
\newcommand{\Z}{\mathbb{Z}}
\newcommand{\N}{\mathbb{N}}
\newcommand{\Q}{\mathbb{Q}}
\newcommand{\F}{\mathbb{F}}
\newcommand{\C}{\mathbb{C}}
\newcommand{\D}{\mathbb{D}}
\newcommand{\bH}{\mathbb{H}}

\newcommand{\abs}[1]{\left| #1 \right|}
\newcommand{\set}[1]{\left\{ #1 \right\}}
\newcommand{\brac}[1]{\left[ #1 \right]}
\newcommand{\paren}[1]{\left( #1 \right)}
\newcommand{\norm}[1]{\left\Vert #1 \right\Vert}
\newcommand{\ol}[1]{\overline{#1}}
\newcommand{\iprod}[2]{\left\langle #1, #2 \right\rangle}


\DeclareMathOperator{\real}{Re}
\DeclareMathOperator{\imag}{Im}

\newcommand{\comp}[2]{#1 \circ #2}

\DeclareMathOperator{\dom}{dom}

\newenvironment{solution}{\begin{proof}[Solution]}{\end{proof}}
\newenvironment{definition}{\begin{proof}[Definition]}{\end{proof}}

% lol, lmao
\linespread{1.1}
\renewcommand{\baselinestretch}{1.17}

% from 117
    % \newenvironment{practice}
    % [2]{\begin{trivlist}
    % \item[\hskip \labelsep {\bfseries     
    % % "debug mode": for working on it
    % % Practice #1, Week #2.
    % % "release mode": for submitting
    % Week #2: Practice #1.
    % }\hskip \labelsep]}{\end{trivlist}}
    
\begin{document}

% 117
\renewcommand{\labelenumi}{(\alph{enumi})}

% ------------------------------------------ %
%                 START HERE                 %
% ------------------------------------------ %

\title{Homework 5} % Replace with appropriate title
\author{Nathaniel Hamovitz\\Math 118A, Ponce, F22}
\date{due 2022-11-06}

\maketitle

\textbf{1. }
Let $(X, d)$ be a metric space. Prove that if $\set{A_\alpha : \alpha \in I}$ is a family of connected subsets of $X$ such that $\displaystyle \bigcap_{\alpha \in I} A_\alpha \ne \emptyset$, then $\displaystyle \bigcup_{\alpha \in I} A_\alpha$ is also connected.

\begin{proof}
    
\end{proof}

\newpage % ---


\textbf{2. }
Rudin, Chapter 3, Exercise 14, parts (a) - (d). If $\set{s_n}$ is a complex sequence, define its arithmetic means $\sigma_n$ by
$$\sigma_n = \frac{s_0 + s_1 + \cdots + s_n}{n + 1}.$$

\begin{enumerate}
    \item 
    If $\lim s_n = s$, prove that $\lim \sigma_n = s$.
    \begin{proof}
        By definition of limit, $\forall \varepsilon > 0, \exists N \in \N, n > N \Rightarrow d(s_n, s) < \varepsilon$. 
        
    \end{proof}


    \item 
    Construct a sequence $\set{s_n}$ which does not converge, although $\lim \sigma_n = 0$.
    \begin{proof}
        
    \end{proof}


    \item
    Can it happen that $s_n > 0$ for all $n$ and that $\lim \sup s_n = \infty$, although $\lim \sigma_n = 0$?
    \begin{proof}
        
    \end{proof}


    \item
    Put $a_n = s_n - s_{n - 1}$, for $n \ge 1$. Show that
    $$s_n - \sigma_n = \frac{1}{n + 1} \sum_{k = 1}^n k a_k.$$

    Assume that $\lim (n a_n) = 0$ and that $\set{\sigma_n}$ converges. Prove that $\set{s_n}$ converges. [This gives a converse of (a), but under the additional assumption that $n a_n \to 0$.]
\end{enumerate}


\newpage % ---

\textbf{3. }
Define $a_1 = \sqrt{2}$ and $a_{n + 1} = \sqrt{2 a_n}$, $n \in \N$. Prove that the sequence $(a_n)_{n = 1}^\infty$ converges and find its limit.

\begin{proof}
    % First we show with induction that $\forall n \in N$, $a_n < 2$. For the base case, clearly $a_1 = \sqrt{2} < 2$. Now suppose that $a_n < 2$. Then $\sqrt{a_n} < \sqrt{2}$. We have $a_{n + 1} = \sqrt{2 a_n} = \sqrt{2} \cdot \sqrt{a_n}$. Because $\sqrt{a_n} < \sqrt{2}$, we see that $a_{n + 1} < \sqrt{2}^2 = 2$. 

    % Now we show that 
    

    Note that 
    \begin{align*}
        a_1 &= \sqrt{2} &= \sqrt{2} &= 2^{\frac{1}{2}}  &= 2^{\frac{1}{2}} \\
        a_2 &= \sqrt{2 \sqrt{2}} &= \sqrt{2} \sqrt{\sqrt{2}} &= 2^{\frac{1}{2}} \cdot 2^{\frac{1}{4}} &= 2^{\frac{1}{2} + \frac{1}{4}} \\
        a_3 &= \sqrt{2\sqrt{2 \sqrt{2}}} &= \sqrt{2} \sqrt{\sqrt{2}} \sqrt{\sqrt{\sqrt{2}}} &= 2^{\frac{1}{2}} \cdot 2^{\frac{1}{4}} \cdot 2^{\frac{1}{8}} &= 2^{\frac{1}{2} + \frac{1}{4} + \frac{1}{8}} \\
        a_n &= 2^{\sum_{j = 1}^n \paren{\frac{1}{2}}^j}
    \end{align*}

    That's a geometric series which sums to $1$, and thus the limit is $2^1 = 2$. I know we haven't learned series yet in this course, but this was my first instinct and I think it's a cool way of doing it.
\end{proof}


\newpage % ---


\textbf{4. }
Let $S = (a_n)_{n = 1}^\infty$ be a bounded real sequence. Let
$$L_S = \set{x \in \R : \text{ exists a subsequence of } S : (a_{n_k})_{k = 1}^\infty \text{ such that } \lim_{k \to \infty} a_{n_k} = x}.$$

Prove:
\begin{enumerate}
    \item 
    $L_S \ne \emptyset$ is closed and bounded.
    \begin{proof}
        First, that $L_S \ne \emptyset$.

        Let $C = \set{a_n : n \in \N}$. We use cases. Case 1: $C$ is finite; that is, $C = \set{p_1, p_2, \dots, p_k}$. Note that $C$ is the image of the naturals under the function $n \mapsto a_n$, and so by pigeonhole principle, $\exists i_0 \in \set{1, \dots, k}$ such that $a_n = p_{i_0}$ for infinitely many $n$. This defines a convergent subsequence, and hence $p_{i_0} \in L_S$; therefore $L_S \ne \emptyset$.

        Case 2: $C$ is infinite. Note that $C$ must be bounded, because $S$ is bounded. That is, $\exists R > 0, C \subseteq B_R(0)$. That ball is compact (by Heine-Borel) and thus $\exists x_0 \in C'$. By definition, $\forall r > 0, B_r(x_0) \cap C - \set{x_0} \ne \emptyset$. That is equivalent to $\forall k \in \N, B_{\frac{1}{k}}(x_0) \cap C - \set{x_0} \ne \emptyset$. (For the forward direction, $\frac{1}{k}$ is a radius greater than $0$; for the backwards, note that $\forall r > 0, \exists k \in \N, r > \frac{1}{k}$.) Now define a subsequence $(a_{n_k})$ such that for each $k$, $a_{n_k}$ is the point in $C$ guaranteed to exists within the ball of radius $\frac{1}{k}$ around $x_0$ by the above. Then $\lim a_{n_k} = x_0$ and hence $x_0 \in L_S$. 

        Step 2: $L_S$ is bounded.

        Since $C$ is bounded, $\exists k > 0, \forall n \in \N, \abs{a_n} \le k$. Take some $x \in L_S$. Then $x = \lim a_{n_j}$ for some subsequence of $S$. Taking the absolute value, $\abs{x} = \abs{\lim a_{n_j}} = \lim \abs{a_{n_j}} \le k$. Thus $x \in \ol{B_k(0)}$ and so $L_S \subseteq \ol{B_k(0)}$, and therefore $L_S$ is bounded.

        Step 3: $L_S$ is closed. If $L_S' = \emptyset$, then $L_S$ is closed. Thus let $x \in L_S'$. Then $\forall r > 0, B_r(x_0) \cap L_S - \set{x_0} \ne \emptyset$. As we saw above, this is equivalent to $\forall n \in \N, B_{\frac{1}{n}}(x_0) \cap L_S - \set{x_0} \ne \emptyset$. That is, $\forall n \in \N, \exists x_n \in L_S, x_n \in B_{\frac{1}{n}}(x_0), x_n \ne x_0$. Now we construct a subsequence of $S$: for each $n$, find $a_{k_n}$ such that $d(x_n, a_{k_n}) < \frac{1}{n}$. Then $\lim a_{n_k} = x_0$ because $d(a_{n_k}, x_0) \le d(a_{k_n}, x_n) + d(x_n, x_0)$. Both the latter distances are less than $\frac{1}{n}$, and so their sum is at most $\frac{2}{n}$
        
    \end{proof}


    \item 
    $L_S$ has only one point ($L_S = \set{x_0}$) if and only if the sequence $\set{a_n}_{n = 1}^\infty$ converges ($\lim_{n \to \infty} a_n = x_0$).
    \begin{proof}

        % ⇒: see pic


        % ⇐ Contrapositive: suppose L_S has more than one point (call two x_0 and y_0). Let d = |x_0 - y_0|. Let \epsilon = d/4. Then there are infinitely many points within epsilon of y_0, which means they are not within epsilon of x_0. 
        % actually contradiction works better: Let lim a_n = x. Suppose for cont that L_s has more than one point. then above. 
        % % Likewise for x_0 and y_0. Thus not convergent anywhere. 
        % Thus x_0 is not the limit. Consider any other point z_0. Let d = distance between x_0 and z_0. Epsilon = d/4 again. Infinitely many close to x_0, so z_0 not a limit. 
        ($\Longleftarrow$). Suppose that the sequence converges. Suppose for a contradiction that $L_S$ has more than one element. Clearly $x_0 \in L_S$; consider the subsequence formed by taking every point in the original sequence. Name one of the other elements $y \in L_S$. Let $d = \abs{x_0 - y}$ and let $\varepsilon = \frac{d}{4}$. By the definition of convergence, $\exists N \in \N$ such that $n_j > N$ implies $|a_{n_j} - y| < \varepsilon$. Then all those $a_{n_j}$ are not within $\varepsilon$ of $x_0$, which means that $\nexists N, n > N \Rightarrow \abs{a_n - x_0} < \varepsilon$. Therefore $x_0$ is not the limit of $(a_n)$, and we have a contradiction.
    \end{proof}







    \item 
    If $\alpha \in L_S$, then
    $$\limsup_{n \to \infty} a_n = M \ge \alpha.$$
    \begin{proof}
        
    \end{proof}


    \item 
    Prove that $M \in L_S$. Hence, $M = \sup L_S$.
    \begin{proof}
        From above, we know the set is closed. Therefore it must contain its supremum. (This is a result from above). 
    \end{proof}

\end{enumerate}
\end{document}

% suggested problems: ch 3, exc 1 2 4 6