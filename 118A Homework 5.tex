\documentclass{article}

\usepackage[margin=1in]{geometry} 
\usepackage{amsmath, amsthm, amssymb, hyperref}

\newcommand{\R}{\mathbb{R}}  
\newcommand{\Z}{\mathbb{Z}}
\newcommand{\N}{\mathbb{N}}
\newcommand{\Q}{\mathbb{Q}}
\newcommand{\F}{\mathbb{F}}
\newcommand{\C}{\mathbb{C}}
\newcommand{\D}{\mathbb{D}}
\newcommand{\bH}{\mathbb{H}}

\newcommand{\abs}[1]{\left| #1 \right|}
\newcommand{\set}[1]{\left\{ #1 \right\}}
\newcommand{\brac}[1]{\left[ #1 \right]}
\newcommand{\paren}[1]{\left( #1 \right)}
\newcommand{\norm}[1]{\left\Vert #1 \right\Vert}
\newcommand{\ol}[1]{\overline{#1}}
\newcommand{\iprod}[2]{\left\langle #1, #2 \right\rangle}


\DeclareMathOperator{\real}{Re}
\DeclareMathOperator{\imag}{Im}

\newcommand{\comp}[2]{#1 \circ #2}

\DeclareMathOperator{\dom}{dom}

\newenvironment{solution}{\begin{proof}[Solution]}{\end{proof}}
\newenvironment{definition}{\begin{proof}[Definition]}{\end{proof}}

% lol, lmao
\linespread{1.1}
\renewcommand{\baselinestretch}{1.17}

% from 117
    % \newenvironment{practice}
    % [2]{\begin{trivlist}
    % \item[\hskip \labelsep {\bfseries     
    % % "debug mode": for working on it
    % % Practice #1, Week #2.
    % % "release mode": for submitting
    % Week #2: Practice #1.
    % }\hskip \labelsep]}{\end{trivlist}}
    
\begin{document}

% 117
\renewcommand{\labelenumi}{(\alph{enumi})}

% ------------------------------------------ %
%                 START HERE                 %
% ------------------------------------------ %

\title{Homework 5} % Replace with appropriate title
\author{Nathaniel Hamovitz\\Math 118A, Ponce, F22}
\date{due 2022-11-06}

\maketitle

\textbf{1. }
Let $(X, d)$ be a metric space. Prove that if $\set{A_\alpha : \alpha \in I}$ is a family of connected subsets of $X$ such that $\displaystyle \bigcap_{\alpha \in I} A_\alpha \ne \emptyset$, then $\displaystyle \bigcup_{\alpha \in I} A_\alpha$ is also connected.

\begin{proof}
    
\end{proof}

\newpage % ---


\textbf{2. }
Rudin, Chapter 3, Exercise 14, parts (a) - (d). If $\set{s_n}$ is a complex sequence, define its arithmetic means $\sigma_n$ by
$$\sigma_n = \frac{s_0 + s_1 + \cdots + s_n}{n + 1}$$

\begin{enumerate}
    \item 
    If $\lim s_n = s$, prove that $\lim \sigma_n = s$.
    \begin{proof}
        
    \end{proof}


    \item 
    Construct a sequence $\set{s_n}$ which does not converge, although $\lim \sigma_n = 0$.
    \begin{proof}
        
    \end{proof}


    \item
    Can it happen that $s_n > 0$ for all $n$ and that $\lim \sup s_n = \infty$, although $\lim \sigma_n = 0$?
    \begin{proof}
        
    \end{proof}


    \item
    Put $a_n = s_n - s_{n - 1}$, for $n \ge 1$. Show that
    $$s_n - \sigma_n = \frac{1}{n + 1} \sum_{k = 1}^n k a_k.$$

    Assume that $\lim (n a_n) = 0$ and that $\set{\sigma_n}$ converges. Prove that $\set{s_n}$ converges. [This gives a converse of (a), but under the additional assumption that $n a_n \to 0$.]
\end{enumerate}


\newpage % ---

\textbf{3. }
Define $a_1 = \sqrt{2}$ and $a_{n + 1} = \sqrt{2 a_n}$, $n \in \N$. Prove that the sequence $(a_n)_{n = 1}^\infty$ converges and find its limit.

\begin{proof}
    First we show with induction that $\forall n \in N$, $a_n < 2$. For the base case, clearly $a_1 = \sqrt{2} < 2$. Now suppose that $a_n < 2$. Then $\sqrt{a_n} < \sqrt{2}$. We have $a_{n + 1} = \sqrt{2 a_n} = \sqrt{2} \cdot \sqrt{a_n}$. Because $\sqrt{a_n} < \sqrt{2}$, we see that $a_{n + 1} < \sqrt{2}^2 = 2$. 

    Now we show that 
    
\end{proof}


\newpage % ---


\textbf{4. }
Let $S = (a_n)_{n = 1}^\infty$ be a bounded real sequence. Let
$$L_S = \set{x \in \R : \text{ exists a subsequence of } S : (a_{n_k})_{k = 1}^\infty \text{ such that } \lim_{k \to \infty} a_{n_k} = x}.$$

Prove:
\begin{enumerate}
    \item 
    $L_S \ne \emptyset$ is closed and bounded.
    \begin{proof}
        
    \end{proof}


    \item 
    $L_S$ has only one point ($L_S = \set{x_0}$) if and only if the sequence $\set{a_n}_{n = 1}^\infty$ converges ($\lim_{n \to \infty} a_n = x_0$).
    \begin{proof}

        % ⇒: see pic


        % ⇐ Contrapositive: suppose L_S has more than one point (call two x_0 and y_0). Let d = |x_0 - y_0|. Let \epsilon = d/4. Then there are infinitely many points within epsilon of y_0, which means they are not within epsilon of x_0. 
        % actually contradiction works better: Let lim a_n = x. Suppose for cont that L_s has more than one point. then above. 
        % % Likewise for x_0 and y_0. Thus not convergent anywhere. 
        % Thus x_0 is not the limit. Consider any other point z_0. Let d = distance between x_0 and z_0. Epsilon = d/4 again. Infinitely many close to x_0, so z_0 not a limit. 
    \end{proof}


    \item 
    If $\alpha \in L_S$, then
    $$\limsup_{n \to \infty} a_n = M \ge \alpha.$$
    \begin{proof}
        
    \end{proof}


    \item 
    Prove that $M \in L_S$. Hence, $M = \sup L_S$.
    \begin{proof}
        From above, we know the set is closed. Therefore it must contain its supremum. (This is a result from above). 
    \end{proof}

\end{enumerate}
\end{document}

% suggested problems: ch 3, exc 1 2 4 6