\documentclass{article}

\usepackage[margin=1in]{geometry} 
\usepackage{amsmath, amsthm, amssymb, hyperref}

\newcommand{\R}{\mathbb{R}}  
\newcommand{\Z}{\mathbb{Z}}
\newcommand{\N}{\mathbb{N}}
\newcommand{\Q}{\mathbb{Q}}
\newcommand{\F}{\mathbb{F}}
\newcommand{\C}{\mathbb{C}}
\newcommand{\D}{\mathbb{D}}
\newcommand{\bH}{\mathbb{H}}

\newcommand{\abs}[1]{\left| #1 \right|}
\newcommand{\set}[1]{\left\{ #1 \right\}}
\newcommand{\brac}[1]{\left[ #1 \right]}
\newcommand{\paren}[1]{\left( #1 \right)}

\DeclareMathOperator{\real}{Re}
\DeclareMathOperator{\imag}{Im}

\newcommand{\comp}[2]{#1 \circ #2}

\DeclareMathOperator{\dom}{dom}

\newenvironment{theorem}[2][Theorem]{\begin{trivlist}
\item[\hskip \labelsep {\bfseries #1}\hskip \labelsep {\bfseries #2.}]}{\end{trivlist}}
\newenvironment{lemma}[2][Lemma]{\begin{trivlist}
\item[\hskip \labelsep {\bfseries #1}\hskip \labelsep {\bfseries #2.}]}{\end{trivlist}}
\newenvironment{exercise}[2][Exercise]{\begin{trivlist}
\item[\hskip \labelsep {\bfseries #1}\hskip \labelsep {\bfseries #2.}]}{\end{trivlist}}
\newenvironment{problem}[2][Problem]{\begin{trivlist}
\item[\hskip \labelsep {\bfseries #1}\hskip \labelsep {\bfseries #2.}]}{\end{trivlist}}
\newenvironment{question}[2][Question]{\begin{trivlist}
\item[\hskip \labelsep {\bfseries #1}\hskip \labelsep {\bfseries #2.}]}{\end{trivlist}}
\newenvironment{corollary}[2][Corollary]{\begin{trivlist}
\item[\hskip \labelsep {\bfseries #1}\hskip \labelsep {\bfseries #2.}]}{\end{trivlist}}

\newenvironment{solution}{\begin{proof}[Solution]}{\end{proof}}
\newenvironment{definition}{\begin{proof}[Definition]}{\end{proof}}

% lol, lmao
\linespread{1.1}
\renewcommand{\baselinestretch}{1.17}

% from 117
    % \newenvironment{practice}
    % [2]{\begin{trivlist}
    % \item[\hskip \labelsep {\bfseries     
    % % "debug mode": for working on it
    % % Practice #1, Week #2.
    % % "release mode": for submitting
    % Week #2: Practice #1.
    % }\hskip \labelsep]}{\end{trivlist}}
    
\begin{document}

% 117
% \renewcommand{\labelenumi}{(\alph{enumi})}

% ------------------------------------------ %
%                 START HERE                 %
% ------------------------------------------ %

\title{Homework 1} % Replace with appropriate title
\author{Nathaniel Hamovitz\\Math 119A, Nagy, F22}
\date{due 2022-10-04}

\maketitle

\begin{problem}{1}
    For each matrix below compute the Jordan normal form.
\end{problem}

\begin{enumerate}
    \item[I] $\mathbf A = \begin{pmatrix}
        2 & -1 \\
        -1 & 2
    \end{pmatrix}$

    \begin{solution}
        % WA
        $\mathbf A = \mathbf M^{-1} \mathbf J \mathbf M$
        where
        $\mathbf M^{-1}
         = \begin{pmatrix}
            1 & -1 \\
            1 & 1
        \end{pmatrix}$,
        $\mathbf 
        J = \begin{pmatrix}
            1 & 0 \\
            0 & 3
        \end{pmatrix}$, and
        $\mathbf 
        M = \frac{1}{2}\begin{pmatrix}
            1 & 1 \\
            -1 & 1 \\
        \end{pmatrix}$.
    \end{solution}


    \item[II] $B = \begin{pmatrix}
        1 & 4 \\
        -1 & -3
    \end{pmatrix}$

    \begin{solution}
        % WA
        $\mathbf A = \mathbf M^{-1} \mathbf J \mathbf M$
        where
        $\mathbf M^{-1}
         = \begin{pmatrix}
            -2 & -1 \\
            1 & 0
        \end{pmatrix}$,
        $\mathbf 
        J = \begin{pmatrix}
            -1 & 1 \\
            0 & -1
        \end{pmatrix}$, and
        $\mathbf 
        M = \begin{pmatrix}
            0 & 1 \\
            -1 & -2 \\
        \end{pmatrix}$.

    \end{solution}



    % \item[III] $C = \begin{pmatrix}
    %     -3 & 2 \\
    %     -2 & -3
    % \end{pmatrix}$
    % \begin{solution}
    % \end{solution}


\end{enumerate}


% ---


\begin{problem}{2}
    Let $A$, $B$, and $C$ be as in Problem 1. Compute the following matrix exponentials.

    Hint: Recall that if $X = M^{-1}JM$, then $\exp(Xt) = M^{-1} \exp(Jt) M$.
\end{problem}

\begin{enumerate}
    \item[I] $\exp(\mathbf At)$
    \begin{solution}
    \begin{align*}
        \exp(\mathbf At) &= \begin{pmatrix}
            1 & -1 \\
            1 & 1
        \end{pmatrix} \exp \paren{\begin{pmatrix}
            1 & 0 \\
            0 & 3
        \end{pmatrix}t}
        \cdot \frac{1}{2} \begin{pmatrix}
            1 & 1 \\
            -1 & 1 \\
        \end{pmatrix} \\
        &= \begin{pmatrix}
            1 & -1 \\
            1 & 1
        \end{pmatrix} \begin{pmatrix}
            e^t & 0 \\
            0 & e^{3t}
        \end{pmatrix}
        \cdot \frac{1}{2} \begin{pmatrix}
            1 & 1 \\
            -1 & 1 \\
        \end{pmatrix} \\
        &= \frac{e^t}{2} \begin{pmatrix}
            e^{2t}+1 & -e^{2t} + 1 \\
            -e^{2t} + 1 & e^{2t}+1
        \end{pmatrix}
    \end{align*}
    \end{solution}

\vspace*{100pt}

    \item[II] $\exp(\mathbf Bt)$
    \begin{solution}
        \begin{align*}
            \exp(\mathbf Bt) &= \begin{pmatrix}
                -2 & -1 \\
                1 & 0
            \end{pmatrix} \exp \paren{\begin{pmatrix}
                -1 & 1 \\
                0 & -1
            \end{pmatrix}t}
            \begin{pmatrix}
                0 & 1 \\
                -1 & 2 \\
            \end{pmatrix} \\
            &= \begin{pmatrix}
                -2 & -1 \\
                1 & 0
            \end{pmatrix}
            \begin{pmatrix}
                e^{-t} & e^{-t} \\
                0 & e^{-t}
            \end{pmatrix}
            \begin{pmatrix}
                0 & 1 \\
                -1 & 2 \\
            \end{pmatrix} \\
            &= e^{-t} \begin{pmatrix}
                3 & 4 \\
                -1 & -1
            \end{pmatrix}
        \end{align*}
        \end{solution}


    % \item[III] $\exp(Ct)$  
    % \begin{solution}
    % \end{solution}


\end{enumerate}

% ---

\begin{problem}{3}
    Let $A$, $B$, and $C$ be as in Problem 1. Solve the following initial value problems. (We learned two techniques to solve, you may use either, but you have to submit a full solution for full credit.)
\end{problem}

\begin{enumerate}
    \item[I] $\vec x'(t) = A \vec x(t)$, and $\vec x(0) = \begin{pmatrix}
        1 \\
        2
    \end{pmatrix}$

    \begin{solution}
    \end{solution}

    \item[II] $\vec y'(t) = B \vec y(t) + \begin{pmatrix}
        1 \\
        -1
    \end{pmatrix}$, and $\vec y(0) = \begin{pmatrix}
        0 \\
        1
    \end{pmatrix}$

    \begin{solution}
    \end{solution}

    

    % \item[III] $\vec z'(t) = \mathbf{C} \vec z(t) + \begin{pmatrix}
    %     t \\
    %     0
    % \end{pmatrix}$, and $\vec z(0) = \begin{pmatrix}
    %     3 \\
    %     -2
    % \end{pmatrix}$

    % \begin{solution}
    % \end{solution}
\end{enumerate}

% ---

\begin{problem}{4}
    Use mathematical induction to prove that for all $n \in \N_+$, we have the following:
\end{problem}

\begin{enumerate}
    \item[I]
    $(M^{-1} X M)^n = M^{-1} X^n M$. (Here $M$ is any invertible matrix and $X$ is any square matrix).
    \begin{proof}
        The statement is clearly true for the base case of $n = 1$.
        
        For the induction step, suppose that for some $n \in \N$, $(M^{-1} X M)^n = M^{-1} X^n M$. Multiply both sides by $M^{-1} X M$ on the right. On the LHS, this is the expression inside the parenthetical, so the exponent becomes $n + 1$. On the RHS, we have $M^{-1} X^n M M^{-1} X M$; the multiplication of $M$ and its inverse in the center cancels out and we have $M^{-1} X^n X M$. Then because $X^n X = X^{n + 1}$ the RHS simplifies to $M^{-1} X^{n + 1} M$.

        Thus, we see that if we assume $(M^{-1} X M)^n = M^{-1} X^n M$, then $(M^{-1} X M)^{n + 1} = M^{-1} X^{n + 1} M$. Hence the relation holds for all $n \in \N$.
    \end{proof}



    \item[II]
    $\begin{pmatrix}
        \lambda & 1 \\
        0 & \lambda
    \end{pmatrix}^n =
    \begin{pmatrix}
        \lambda^n & n \lambda^{n - 1} \\
        0 & \lambda^n
    \end{pmatrix}$
    \begin{proof}
        The base case $n = 1$ is evident:
        $$\begin{pmatrix}
            \lambda & 1 \\
            0 & \lambda
        \end{pmatrix}^1 =
        \begin{pmatrix}
            \lambda & 1 \\
            0 & \lambda
        \end{pmatrix} =
        \begin{pmatrix}
            \lambda^1 & 1 \cdot \lambda^{0} \\
            0 & \lambda^1
        \end{pmatrix}$$

        For the induction step, suppose that for some $n \in \N$,
        $$\begin{pmatrix}
            \lambda & 1 \\
            0 & \lambda
        \end{pmatrix}^n =
        \begin{pmatrix}
            \lambda^n & n \lambda^{n - 1} \\
            0 & \lambda^n
        \end{pmatrix}.$$

        Multiply on the right by the matrix $\begin{pmatrix}
            \lambda & 1 \\
            0 & \lambda
        \end{pmatrix}$. On the left hand side, this just increses the exponent by one and we have $\begin{pmatrix}
            \lambda & 1 \\
            0 & \lambda
        \end{pmatrix}^{n + 1}$. On the right hand side, we have
        $$\begin{pmatrix}
            \lambda^n & n \lambda^{n - 1} \\
            0 & \lambda^n
        \end{pmatrix}
        \begin{pmatrix}
            \lambda & 1 \\
            0 & \lambda
        \end{pmatrix}
        = \begin{pmatrix}
            \lambda^n \cdot \lambda + n \lambda^{n - 1} \cdot 0 & \lambda^n \cdot 1 + n \lambda^{n - 1} \cdot \lambda \\
            0 \cdot \lambda + \lambda^n \cdot 0 & 0 \cdot 1 + \lambda^n \cdot \lambda
        \end{pmatrix}
        = \begin{pmatrix}
            \lambda^{n + 1} & (n + 1)\lambda^{n} \\
            0 & \lambda^{n + 1}
        \end{pmatrix}$$

        which implies that the proposition holds for $n + 1$, and therefore for all $n \in \N$.   
        
    \end{proof}


\end{enumerate}

\end{document}