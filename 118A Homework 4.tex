\documentclass{article}

\usepackage[margin=1in]{geometry} 
\usepackage{amsmath, amsthm, amssymb, hyperref}

\newcommand{\R}{\mathbb{R}}  
\newcommand{\Z}{\mathbb{Z}}
\newcommand{\N}{\mathbb{N}}
\newcommand{\Q}{\mathbb{Q}}
\newcommand{\F}{\mathbb{F}}
\newcommand{\C}{\mathbb{C}}
\newcommand{\D}{\mathbb{D}}
\newcommand{\bH}{\mathbb{H}}

\newcommand{\abs}[1]{\left| #1 \right|}
\newcommand{\set}[1]{\left\{ #1 \right\}}
\newcommand{\brac}[1]{\left[ #1 \right]}
\newcommand{\paren}[1]{\left( #1 \right)}
\newcommand{\norm}[1]{\left\Vert #1 \right\Vert}
\newcommand{\ol}[1]{\overline{#1}}
\newcommand{\iprod}[2]{\left\langle #1, #2 \right\rangle}


\DeclareMathOperator{\real}{Re}
\DeclareMathOperator{\imag}{Im}

\newcommand{\comp}[2]{#1 \circ #2}

\DeclareMathOperator{\dom}{dom}

\newenvironment{theorem}[2][Theorem]{\begin{trivlist}
\item[\hskip \labelsep {\bfseries #1}\hskip \labelsep {\bfseries #2.}]}{\end{trivlist}}
\newenvironment{lemma}[2][Lemma]{\begin{trivlist}
\item[\hskip \labelsep {\bfseries #1}\hskip \labelsep {\bfseries #2.}]}{\end{trivlist}}
\newenvironment{exercise}[2][Exercise]{\begin{trivlist}
\item[\hskip \labelsep {\bfseries #1}\hskip \labelsep {\bfseries #2.}]}{\end{trivlist}}
\newenvironment{problem}[2][Problem]{\begin{trivlist}
\item[\hskip \labelsep {\bfseries #1}\hskip \labelsep {\bfseries #2.}]}{\end{trivlist}}
\newenvironment{question}[2][Question]{\begin{trivlist}
\item[\hskip \labelsep {\bfseries #1}\hskip \labelsep {\bfseries #2.}]}{\end{trivlist}}
\newenvironment{corollary}[2][Corollary]{\begin{trivlist}
\item[\hskip \labelsep {\bfseries #1}\hskip \labelsep {\bfseries #2.}]}{\end{trivlist}}

\newenvironment{solution}{\begin{proof}[Solution]}{\end{proof}}
\newenvironment{definition}{\begin{proof}[Definition]}{\end{proof}}

% lol, lmao
\linespread{1.1}
\renewcommand{\baselinestretch}{1.17}

% from 117
    % \newenvironment{practice}
    % [2]{\begin{trivlist}
    % \item[\hskip \labelsep {\bfseries     
    % % "debug mode": for working on it
    % % Practice #1, Week #2.
    % % "release mode": for submitting
    % Week #2: Practice #1.
    % }\hskip \labelsep]}{\end{trivlist}}
    
\begin{document}

% 117
\renewcommand{\labelenumi}{(\alph{enumi})}

% ------------------------------------------ %
%                 START HERE                 %
% ------------------------------------------ %

\title{Homework 4} % Replace with appropriate title
\author{Nathaniel Hamovitz\\Math 118A, Ponce, F22}
\date{due 2022-10-25}

\maketitle

\textbf{1. }
Consider $\R$ with the usual distance. Using only the definition, prove

\begin{enumerate}
    \item  % ✅
    $A = [0, 1)$ is not compact.
    \begin{proof}
        Consider the open cover $\set{\sigma_n}_{n \in \N}$, where $\sigma_n = \paren{-\frac{1}{2}, 1 - \frac{1}{n}}$. As the infimum of each interval remains constant and the supremum is increasing we have that $\forall n \in \N, \sigma_n \subsetneq \sigma_{n + 1}$. 
        
        Given any finite subcover $\set{\sigma_{n_i}}$, we can consider $j = \max \set{n_i}$. From above, we see that $\cup_{n_i} \sigma_{n_i} \subseteq \sigma_j$. But there must exist $x \in [0, 1)$ such that $x \notin \sigma_j$; consider $x = 1 - \frac{1}{2j}$. 

        Hence there is at least one open cover of $A$ from which it is not possible to extract a finite subcover, and so $A$ is not compact.
    \end{proof}


    \item  % ✅ but could be touched up
    $D = \set{1 + \frac{(-1)^{n}}{n} : n \in \N} \cup \set{1}$ is compact.
    \begin{proof}
        Let $\set{\sigma_alpha}_{\alpha \in I}$ be an open covering of $D$. Then there must be some open set $\sigma_0$ such that $1 \in \sigma_0$. 

        $\sigma_0$ contains all but finitely many elements of $D$; each remaining element must be contained in some set $\sigma_n$. There are finitely many, so considering those $\sigma_n$ along with $\sigma_0$ gives us a finite subcovering of $D$.        
    \end{proof}
\end{enumerate}


\newpage % ---


\textbf{2. }
Prove that if $A_1, \dots, A_n, \dots$ is a countable collection of countable sets, then
$$\bigcup_{n = 1}^{\infty} A_n \qquad \text{is also countable.}$$

\begin{proof}
    % line em up and diagonals!
    
\end{proof}


\newpage % ---


\textbf{3. }
Consider $\R^n$ with the usual distance. Prove that if $A \subset
% ❔: possibly-equal or strict subset?
\R^n$ is uncountable, then $A' \ne \emptyset$ (i.e. the set of limit points of $A$ is nonempty).

\begin{proof}
    % Malik did this in OH Day 4-2, starting from the version of B-W Ponce proved in class

    First we place an open ball of radius $1$ around every rational. 

\end{proof}


\newpage % ---


\textbf{4. }
Consider $\R^2$ with the usual distance, and define
$$E = \set{\paren{\sin(n), \cos(m)} : n, m \in \Z}$$

Prove that $E' \ne \emptyset$.

\begin{proof}
    Let $B = [-1, 1] \times [-1, 1]$. Clearly $E \subsetneq B$. $B$ is closed and bounded in $\R^2$ and so by Heine-Borel compact.

    Consider the function $f : \N \mapsto \R^2$ where $f(n) = \paren{\sin n, \cos 0} = \paren{\sin n, 1}$. $f$ is an injective map because $\sin a = \sin b$ iff $a - b = 2 \pi k$ for some $k \in \Z$, and no natural numbers are a multiple of $2\pi$ apart (if two were, then $\pi$ could be written as a rational number). Therefore $f[\N]$ is infinite. Now notice that $f[\N] \subsetneq E$, and hence $E$ is infinite.

    We now apply the Bolzano-Weierstrass theorem. $E$ is an infinite set and a subset of a compact set, and therefore $E' \ne \emptyset$.    
\end{proof}


\newpage % ---


\textbf{5. }
Let $K_1, \dots, K_n, \dots$ be a family of compact subsets of $\R^n$ considered with the usual distance.
\begin{enumerate}
    \item 
    Is $K_1 \cup K_2$ compact?
    \begin{proof}
        
    \end{proof}


    \item 
    Is $\bigcup_{n = 1}^\infty K_n$ compact?
    \begin{proof}
        
    \end{proof}


    \item 
    Is $\bigcap_{n = 1}^\infty K_n$ compact?
    \begin{proof}
        
    \end{proof}
\end{enumerate}

% suggested exercises: Rudin Ch 2, 19 and 22




\end{document}