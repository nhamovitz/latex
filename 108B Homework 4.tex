\documentclass{article}

\usepackage[margin=1in]{geometry} 
\usepackage{amsmath, amsthm, amssymb, hyperref}

\newcommand{\R}{\mathbb{R}}  
\newcommand{\Z}{\mathbb{Z}}
\newcommand{\N}{\mathbb{N}}
\newcommand{\Q}{\mathbb{Q}}
\newcommand{\F}{\mathbb{F}}
\newcommand{\C}{\mathbb{C}}
\newcommand{\D}{\mathbb{D}}
\newcommand{\bH}{\mathbb{H}}

\newcommand{\abs}[1]{\left| #1 \right|}
\newcommand{\set}[1]{\left\{ #1 \right\}}
\newcommand{\brac}[1]{\left[ #1 \right]}
\newcommand{\paren}[1]{\left( #1 \right)}
\newcommand{\norm}[1]{\left\Vert #1 \right\Vert}
\newcommand{\ol}[1]{\overline{#1}}
\newcommand{\iprod}[2]{\left\langle #1, #2 \right\rangle}
\newcommand{\lmap}[1]{\mathcal{L}\paren{#1}}
\newcommand{\lmapp}[2]{\mathcal{L}\paren{#1, #2}}


\DeclareMathOperator{\real}{Re}
\DeclareMathOperator{\imag}{Im}

\newcommand{\comp}[2]{#1 \circ #2}

\DeclareMathOperator{\dom}{dom}
\DeclareMathOperator{\range}{range}

\newenvironment{solution}{\begin{proof}[Solution]}{\end{proof}}
\newenvironment{definition}{\begin{proof}[Definition]}{\end{proof}}

% lol, lmao
\linespread{1.1}
\renewcommand{\baselinestretch}{1.17}

% from 117
    % \newenvironment{practice}
    % [2]{\begin{trivlist}
    % \item[\hskip \labelsep {\bfseries     
    % % "debug mode": for working on it
    % % Practice #1, Week #2.
    % % "release mode": for submitting
    % Week #2: Practice #1.
    % }\hskip \labelsep]}{\end{trivlist}}
    
\begin{document}

% 117
% \renewcommand{\labelenumi}{(\alph{enumi})}

% ------------------------------------------ %
%                 START HERE                 %
% ------------------------------------------ %

\title{Homework 4} % Replace with appropriate title
\author{Nathaniel Hamovitz\\Math 108B, Sung, F22}
\date{due 2022-11-10}

\maketitle

% Axler 2e, Ch 7, Problems 4, 21, 23, 29, 30

\textbf{1. }
Fix a field $k$. Given $k$-vector spaces $U$, $V$, and $W$, and $k$-linear maps $f: U \mapsto V$ and $G: V \mapsto W$, the diagram
$$0 \longrightarrow U \longrightarrow^{f}$$

% ---

\textbf{Axler 2e, Exercise 7.4}
Suppose $P \in \lmap{V}$ is such that $P^2 = P$. Prove that $P$ is an orthogonal projection iff $P$ is self-adjoint.
\begin{proof}
    
\end{proof}

% ---

\textbf{Axler 2e, Exercise 7.21}
Prove or give a counterexample: if $S \in \lmap{V}$ and there exists an orthogonal basis $(e_1, \dots, e_n)$ of $V$ such that $\norm{S e_j} = 1$ for each $e_j$, then $S$ is an isometry.
\begin{proof}%[Counterexample]
    
\end{proof}

% ---

\textbf{Axler 2e, Exercise 7.23}
Define $T \in \lmap{\F^3}$ by
$$T(z_1, z_2, z_3) = (z_3, 2_z1, 3z_2).$$

Find (explicitly) an isometry $S \in \lmap{\F^3}$ such that $T = S\sqrt{T^* T}$.

\begin{proof}[Consider]
    
\end{proof}

% ---

\textbf{Axler 2e, Exercise 7.29}
Suppose $T \in \lmap{V}$. Prove that $\dim \range T$ equals the number of nonzero singular values of $T$.

\begin{proof}
    
\end{proof}

% ---

\textbf{Axler 2e, Exercise 7.30}
Suppose $S \in \lmap{V}$. Prove that $S$ is an isometry iff all the singular values of $S$ equal $1$.

\begin{proof}
    
\end{proof}



\end{document}