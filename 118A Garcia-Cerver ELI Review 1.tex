\documentclass{article}

\usepackage[margin=1in]{geometry} 
\usepackage{amsmath,amsthm,amssymb,hyperref}

\newcommand{\R}{\mathbb{R}}  
\newcommand{\Z}{\mathbb{Z}}
\newcommand{\N}{\mathbb{N}}
\newcommand{\Q}{\mathbb{Q}}
\newcommand{\F}{\mathbb{F}}
\newcommand{\C}{\mathbb{C}}
\newcommand{\D}{\mathbb{D}}
\newcommand{\bH}{\mathbb{H}}

\newcommand{\abs}[1]{\left| #1 \right|}
\newcommand{\set}[1]{\left\{ #1 \right\}}
\newcommand{\brac}[1]{\left[ #1 \right]}
\newcommand{\paren}[1]{\left( #1 \right)}

\DeclareMathOperator{\real}{Re}
\DeclareMathOperator{\imag}{Im}

\newcommand{\comp}[2]{#1 \circ #2}

\DeclareMathOperator{\dom}{dom}

\newenvironment{theorem}[2][Theorem]{\begin{trivlist}
\item[\hskip \labelsep {\bfseries #1}\hskip \labelsep {\bfseries #2.}]}{\end{trivlist}}
\newenvironment{lemma}[2][Lemma]{\begin{trivlist}
\item[\hskip \labelsep {\bfseries #1}\hskip \labelsep {\bfseries #2.}]}{\end{trivlist}}
\newenvironment{exercise}[2][Exercise]{\begin{trivlist}
\item[\hskip \labelsep {\bfseries #1}\hskip \labelsep {\bfseries #2.}]}{\end{trivlist}}
\newenvironment{problem}[2][Problem]{\begin{trivlist}
\item[\hskip \labelsep {\bfseries #1}\hskip \labelsep {\bfseries #2.}]}{\end{trivlist}}
\newenvironment{question}[2][Question]{\begin{trivlist}
\item[\hskip \labelsep {\bfseries #1}\hskip \labelsep {\bfseries #2.}]}{\end{trivlist}}
\newenvironment{corollary}[2][Corollary]{\begin{trivlist}
\item[\hskip \labelsep {\bfseries #1}\hskip \labelsep {\bfseries #2.}]}{\end{trivlist}}

\newenvironment{solution}{\begin{proof}[Solution]}{\end{proof}}
\newenvironment{definition}{\begin{proof}[Definition]}{\end{proof}}

% lol, lmao
\linespread{1.1}
\renewcommand{\baselinestretch}{1.17}

% from 117
    % \newenvironment{practice}
    % [2]{\begin{trivlist}
    % \item[\hskip \labelsep {\bfseries     
    % % "debug mode": for working on it
    % % Practice #1, Week #2.
    % % "release mode": for submitting
    % Week #2: Practice #1.
    % }\hskip \labelsep]}{\end{trivlist}}
    
\begin{document}

% 117
% \renewcommand{\labelenumi}{(\alph{enumi})}

% ------------------------------------------ %
%                 START HERE                 %
% ------------------------------------------ %

\title{ELI Review 1} % Replace with appropriate title
\author{Nathaniel Hamovitz\\Math 118A, Garcia-Cervera, F22}
\date{due 2022-09-29}

\maketitle

{\bfseries{Problem.} }
If $r$ is rational ($r \ne 0$), and $x$ is irrational, prove that $r + x$ and $rx$ are irrational.
\begin{proof}
    Since $r$ is rational, we can write $r = \frac{m}{n}$ where $m, n \in \Z$.

    We prove first that $rx$ is irrational. Suppose for a contradiction that $rx$ is rational. Then we can write $\frac{a}{b} = rx = \frac{m}{n}x$, where $a, b \in \Z$. Then, dividing by $r$, we have $\frac{a}{br} = x$, or $\frac{an}{bm} = x$. $a, n, b, m \in \Z$, so $an \in \Z$ and $bm \in Z$. Now we have $x$ as a quotient of integers, which is a contradiction because $x$ is irrational by hypothesis.

    We now show that $r + x$ is irrational; the proof is similar. Suppose for a contradiction that $r + x$ is rational. Then we can write $\frac{a}{b} = r + x = \frac{m}{n} + x$. We subtract $r$ to find $\frac{a}{b} - \frac{m}{n} = x$, or (finding a common denominator) $x = \frac{an - mb}{bn}$. Again, $an - mb \in \Z$ and $bn \in \Z$, and so we have $x$ as a quotient of integers, which is a contradiction because $x$ is irrational by hypothesis.
\end{proof}

\end{document}