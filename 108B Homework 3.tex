\documentclass{article}

\usepackage[margin=1in]{geometry} 
\usepackage{amsmath, amsthm, amssymb, hyperref}

\newcommand{\R}{\mathbb{R}}  
\newcommand{\Z}{\mathbb{Z}}
\newcommand{\N}{\mathbb{N}}
\newcommand{\Q}{\mathbb{Q}}
\newcommand{\F}{\mathbb{F}}
\newcommand{\C}{\mathbb{C}}
\newcommand{\D}{\mathbb{D}}
\newcommand{\bH}{\mathbb{H}}

\newcommand{\abs}[1]{\left| #1 \right|}
\newcommand{\set}[1]{\left\{ #1 \right\}}
\newcommand{\brac}[1]{\left[ #1 \right]}
\newcommand{\paren}[1]{\left( #1 \right)}
\newcommand{\norm}[1]{\left\Vert #1 \right\Vert}
\newcommand{\ol}[1]{\overline{#1}}
\newcommand{\iprod}[2]{\left\langle #1, #2 \right\rangle}


\DeclareMathOperator{\real}{Re}
\DeclareMathOperator{\imag}{Im}

\newcommand{\comp}[2]{#1 \circ #2}

\DeclareMathOperator{\dom}{dom}
\DeclareMathOperator{\range}{range}
\DeclareMathOperator{\nulle}{null}

\newenvironment{theorem}[2][Theorem]{\begin{trivlist}
\item[\hskip \labelsep {\bfseries #1}\hskip \labelsep {\bfseries #2.}]}{\end{trivlist}}
\newenvironment{lemma}[2][Lemma]{\begin{trivlist}
\item[\hskip \labelsep {\bfseries #1}\hskip \labelsep {\bfseries #2.}]}{\end{trivlist}}
\newenvironment{exercise}[2][Exercise]{\begin{trivlist}
\item[\hskip \labelsep {\bfseries #1}\hskip \labelsep {\bfseries #2.}]}{\end{trivlist}}
\newenvironment{problem}[2][Problem]{\begin{trivlist}
\item[\hskip \labelsep {\bfseries #1}\hskip \labelsep {\bfseries #2.}]}{\end{trivlist}}
\newenvironment{question}[2][Question]{\begin{trivlist}
\item[\hskip \labelsep {\bfseries #1}\hskip \labelsep {\bfseries #2.}]}{\end{trivlist}}
\newenvironment{corollary}[2][Corollary]{\begin{trivlist}
\item[\hskip \labelsep {\bfseries #1}\hskip \labelsep {\bfseries #2.}]}{\end{trivlist}}

\newenvironment{solution}{\begin{proof}[Solution]}{\end{proof}}
\newenvironment{definition}{\begin{proof}[Definition]}{\end{proof}}

% lol, lmao
\linespread{1.1}
\renewcommand{\baselinestretch}{1.17}

% from 117
    % \newenvironment{practice}
    % [2]{\begin{trivlist}
    % \item[\hskip \labelsep {\bfseries     
    % % "debug mode": for working on it
    % % Practice #1, Week #2.
    % % "release mode": for submitting
    % Week #2: Practice #1.
    % }\hskip \labelsep]}{\end{trivlist}}
    
\begin{document}

% 117
\renewcommand{\labelenumi}{(\alph{enumi})}

% ------------------------------------------ %
%                 START HERE                 %
% ------------------------------------------ %

\title{Homework 3} % Replace with appropriate title
\author{Nathaniel Hamovitz\\Math 108B, Sung, F22}
\date{due 2022-10-31}

\maketitle

% Axler second edition, chapter 6: 19, 25, 26, 30, 31.

\textbf{Axler 2e, Exercise 6.19: } % ✅
Suppose $T \in \mathcal{L}(V)$ and $U$ is a subspace of $V$. Prove that $U$ is invariant under $T$ iff $P_U T P_U = T P_U$.

\begin{proof}
    ($\Longrightarrow$) Suppose that $U$ is invariant under $T$. Then $\forall u \in U, Tu \in U$. Let $v \in V$. Then $v = u + u'$ where $u \in U$ and $u' \in U^\perp$. By the definition of the orthogonal projection, $P_U v = u$. Because $U$ is invariant under $T$, $T u = u$. And because $u \in U$, $P_U$ is the identity. Thus the action of $P_U T P_U$ and $T P_U$ is the same for all $v \in V$. 

    ($\Longleftarrow$) Suppose that $P_U T P_U  = T P_U$. 
    % my attempt
    % Then $P_U T = T$. For all $v = u + u' \in V = U \oplus U^\perp$, $P_U v = u$. Hence $P_U (Tv) = P_U(Tu + T u')$ and also 
    Let $u \in U$. Then $P_U u = u \in U$ and so $P_U(T u) = Tu$. Hence $Tu \in U$, and therefore $U$ is invariant under $T$.    
\end{proof}


 % ---


\textbf{Axler 2e, Exercise 6.25: } % ✅
Find a polynomial $q \in \mathcal{P}_2(\R)$ such that
$$\int_0^1 p(x) \cos(\pi x) \: dx = \int_0^1 p(x) q(x) \: dx$$
for every $p \in \mathcal{P}_2(\R)$.

\begin{proof}
    % 3e Thm 6.42 and Ex 6.44, p 188
    Because of the properties of the integral, $\phi : \mathcal{P}_2(\R) \mapsto \R$ defined by $\phi(p) = \int_0^1 p(x) \cos(\pi x) \: dx$ is a linear functional on $\mathcal{P}_2(\R)$. Therefore, by the Riesz Representation Theorem, there must exist some vector (polynomial) $q \in \mathcal{P}_2(\R)$ such that $\phi(p) = \iprod{p}{q}$. We can define the inner product as the integral from 0 to 1 of the product of the two polynomials; now the question asks us to find the $q$ which Riesz says must exist. 

    Luckily, Riesz also tells us a formula for such a $q$! Starting from an orthonormal basis $e_1, \dots, e_n$, we have $q = \ol{\phi(e_1)} e_1 + \cdots + \ol{\phi(e_n)}e_n$; as we are dealing only with the reals here, we simplify to
    $$q = \phi(e_1)e_1 + \cdots + \phi(e_n) e_n.$$

    We found an orthonormal basis for $\mathcal{P}_2(\R)$ in the previous homework:
    \begin{align*}
        e_1 &= 1 \\
        e_2 &= \sqrt{3}(2x - 1) \\
        e_2 &= \sqrt{5}(6x^2 - 6x + 1)
    \end{align*}

    The final answer is
    $$q(x) = \frac{12 - 24x}{\pi^2}$$
\end{proof}


 % ---


\textbf{Axler 2e, Exercise 6.26: } % ✅
Fix a vector $v \in V$ and define $T \in \mathcal{L}(V, \F)$ by $Tu = \iprod{u}{v}$. For $a \in \F$, find a formula for $T^* a$.

\begin{proof}
    $T^* \in \mathcal{L}(\F, V)$ is defined as the operator such that $\iprod{Tu}{a} = {u, T^* a}$. We manipulate:
    \begin{align*}
        \iprod{Tu}{a} &= \iprod{u}{T^* a} \\
        \iprod{\iprod{u}{v}}{a} &= \\
        \iprod{u}{v}\ol{a} &= \\
        \iprod{\ol{a}u}{v} &= \\
        \ol{a} \iprod{u}{v} &= \\
        \iprod{u}{av} &= \iprod{u}{T^* a}
    \end{align*}
    and so $T^* a = av$.    
\end{proof}


 % ---


\textbf{Axler 2e, Exercise 6.30: } % ✅
Suppose $T \in \mathcal{L}(V, W)$. Prove that

\begin{enumerate}
    \item 
    $T$ is injective iff $T^*$ is surjective.
    \begin{proof}
        $T$ is injective iff $\nulle T = \set{0}$ (by Axler 3e Thm 3.16). But $\nulle T = (\range T^*)^\perp$, so we have $(\range T^*)^\perp = \set{0}$. That is the case iff $\range T^* = W$, which is equivalent to $T^*$ being surjective.
        Therefore $T^*$ is surjective.
    \end{proof}


    \item 
    $T$ is surjective iff $T*$ is injective.
    \begin{proof}
        Replace $T$ above with $T^*$.        
    \end{proof}
\end{enumerate}


 % ---


\textbf{Axler 2e, Exercise 6.31: }
For every $T \in \mathcal{L}(V, W)$, prove that
$$\dim \nulle T^* = \dim \nulle T + \dim W - \dim V$$
\begin{proof}
    Note that by the Fundamental Theorem of Linear Maps (Axler 3e Thm 3.22), $\dim \nulle T* = \dim W - \dim \range T^*$. Thus it remains to show that $\dim \range T^* = \dim V - \dim \nulle T$; to do so, notice that $\range T^* = (\nulle T)^\perp$ and clearly $\dim (\nulle T)^\perp = \dim V - \dim \nulle T$.
\end{proof}

\textbf{and}
$$\dim \range T^* = \dim \range T.$$
\begin{proof}
    \begin{align*}
        \dim V - \dim \nulle T &= \dim \range T \\
        \dim (\nulle T)^\perp &= \dim \range T \\
        \dim \range T^* &= \dim \range T
    \end{align*}    
\end{proof}


\end{document}