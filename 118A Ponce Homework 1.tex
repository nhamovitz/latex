\documentclass{article}

\usepackage[margin=1in]{geometry} 
\usepackage{amsmath, amsthm, amssymb, hyperref}

\newcommand{\R}{\mathbb{R}}  
\newcommand{\Z}{\mathbb{Z}}
\newcommand{\N}{\mathbb{N}}
\newcommand{\Q}{\mathbb{Q}}
\newcommand{\F}{\mathbb{F}}
\newcommand{\C}{\mathbb{C}}
\newcommand{\D}{\mathbb{D}}
\newcommand{\bH}{\mathbb{H}}

\newcommand{\abs}[1]{\left| #1 \right|}
\newcommand{\set}[1]{\left\{ #1 \right\}}
\newcommand{\brac}[1]{\left[ #1 \right]}
\newcommand{\paren}[1]{\left( #1 \right)}

\DeclareMathOperator{\real}{Re}
\DeclareMathOperator{\imag}{Im}

\newcommand{\comp}[2]{#1 \circ #2}

\DeclareMathOperator{\dom}{dom}

\newenvironment{theorem}[2][Theorem]{\begin{trivlist}
\item[\hskip \labelsep {\bfseries #1}\hskip \labelsep {\bfseries #2.}]}{\end{trivlist}}
\newenvironment{lemma}[2][Lemma]{\begin{trivlist}
\item[\hskip \labelsep {\bfseries #1}\hskip \labelsep {\bfseries #2.}]}{\end{trivlist}}
\newenvironment{exercise}[2][Exercise]{\begin{trivlist}
\item[\hskip \labelsep {\bfseries #1}\hskip \labelsep {\bfseries #2.}]}{\end{trivlist}}
\newenvironment{problem}[2][Problem]{\begin{trivlist}
\item[\hskip \labelsep {\bfseries #1}\hskip \labelsep {\bfseries #2.}]}{\end{trivlist}}
\newenvironment{question}[2][Question]{\begin{trivlist}
\item[\hskip \labelsep {\bfseries #1}\hskip \labelsep {\bfseries #2.}]}{\end{trivlist}}
\newenvironment{corollary}[2][Corollary]{\begin{trivlist}
\item[\hskip \labelsep {\bfseries #1}\hskip \labelsep {\bfseries #2.}]}{\end{trivlist}}

\newenvironment{solution}{\begin{proof}[Solution]}{\end{proof}}
\newenvironment{definition}{\begin{proof}[Definition]}{\end{proof}}

% lol, lmao
\linespread{1.1}
\renewcommand{\baselinestretch}{1.17}

% from 117
    % \newenvironment{practice}
    % [2]{\begin{trivlist}
    % \item[\hskip \labelsep {\bfseries     
    % % "debug mode": for working on it
    % % Practice #1, Week #2.
    % % "release mode": for submitting
    % Week #2: Practice #1.
    % }\hskip \labelsep]}{\end{trivlist}}
    
\begin{document}

% 117
% \renewcommand{\labelenumi}{(\alph{enumi})}

% ------------------------------------------ %
%                 START HERE                 %
% ------------------------------------------ %

\title{Homework 1} % Replace with appropriate title
\author{Nathaniel Hamovitz\\Math 118A, Ponce, F22}
\date{due 2022-09-30}

\maketitle

\begin{problem}{1}
Prove that if $n \in \N$, then $\sqrt{n} \in \N \cup (\Q)^c$. Hint: Review and use the unique-prime-factorization theorem.
\end{problem}
\begin{proof}
    We know by the unique prime factorization that $n = p_1^{k_1} p_2^{k_2} \cdots p_m^{k_m}$, where $p_i$ are distinct primes and $k_i \in \N$. If all $k_i$ are even, then $\sqrt{n} = p_1^{m_1} p_2^{m_2} \cdots p_n^{m_n}$, where $m_i = \frac{k_i}{2} \in \N$, and it is evident that $\sqrt{n} \in \N$.

    If at least one $k_i$ is odd, then $\sqrt{n}$ must be irrational.
    % needed?
    % Without loss of generality (multiplication commutes), assume $k_1$ through $k_o$ are odd and $k_{o + 1}$ through $k_m$ are even, and let $s = p_1 p_2 \cdots p_o$.
    Without loss of generality (multiplication commutes), assume $k_1$ is odd. It suffices to show that $\sqrt{p_1}$ is irrational, because the product of an irrational number is irrational.
    
    Assume for a contradiction that $\sqrt{p_1} \in \Q$. Then we can write $\sqrt{n} = \sqrt{p_1}= \frac{a}{b}$, where $a, b \in \Z$. Without loss of generality, we assume $a$ and $b$ are coprime. Now we have $\paren{\frac{a}{b}}^2 = p_1$. Then $a^2 = b^2 p_1$, which implies $p_1 \mid a^2$. Recall that for any prime $p$ and square number $t^2$, $p \mid t^2 \Rightarrow p \mid t$. Thus we have $p_1 \mid a$, and thus $p_1^2 \mid a^2$, and thus (referring to the original equation) $p_1^2 \mid b^2 p_1$, which implies $p_1 \mid b^2$, and therefore $p_1 \mid b$. This is a contradiction, as we assuemd $a$ and $b$ were coprime. Hence $\sqrt{p_1} \in (\Q)^c$, and so $\sqrt{n} = \sqrt{p_1} \sqrt{p_1^{k_1 - 1} p_2^{k_2} \cdots p_m^{k_m}} \in (\Q)^c$.

    Therefore $\sqrt{n} \in \N \cup (\Q)^c$.
\end{proof}

% --

\begin{problem}{2}
    Find an example of a set $A \subset \R$ such that $\lambda = \sup A$ is well defined but we may not be able to find it explicitly.
\end{problem}

\begin{solution}
    One example is the set $A = \{\cos(n) : n \in \N\}$. $A$ is bounded, and so by the least upper bound property must have a supremum, but it is unclear what the supremum is.
\end{solution}

% ---

\begin{problem}{3}
    Rudin, Exercise 1-16: Suppose $k \ge 3$, $\hat x$, $\hat y \in \R^k$, $\abs{\hat x - \hat y} = d > 0$, and $r > 0$. Prove: 
\end{problem}

\begin{enumerate}
    \item[(a)] If $2r > d$, there are infinitely many $\hat z \in \R^k$ such that     
    $$\abs{\hat z - \hat x} = \abs{\hat z - \hat y} = r$$

    \begin{proof}
        Without loss of generality (all the relevant operations are invariant under translation and rotation), take $\hat x$ and $\hat y$ to lay along the $x$-axis, equidistant from the origin. That is, $\hat x = \paren{-\frac{d}{2}, 0, \dots, 0}$ and $\hat y = \paren{\frac{d}{2}, 0, \dots, 0}$.

        I claim that the intersection is a sphere in dimension $k - 1$, the radius of which is given by $A = \sqrt{r^2 - \paren{\frac{d}{2}}^2}$. The value of $A$ is derived from the Pythagorean theorem in 3 dimensions, starting from $\paren{\frac{d}{2}}^2 + A^2 = r^2$. 

        The curve $C = \{\paren{0, A \cos \theta, A \sin \theta, 0, \dots, 0} : \theta \in [0, 2\pi)\}$ is clearly infinite, and lies along the intersection mentioned above. Thus it suffices to show that $C \subseteq \{z \in \R^k : \abs{\hat x - \hat z} = \abs{\hat y - \hat z} = r\}$.
        
        Let $\hat z \in C$. Then $\exists \theta \in [0, 2\pi)$ such that $\hat z = \paren{0, A \cos \theta, A \sin \theta, 0, \dots, 0}$. We have:

        $$\abs{\hat z - \hat x}^2 = \abs{\begin{pmatrix}
            0 \\
            A \cos \theta \\
            A \sin \theta \\
            \vdots \\
            0 
            \end{pmatrix} - \begin{pmatrix}
                -\frac{d}{2} \\
                0 \\
                0 \\
                0 \\
                0 
                \end{pmatrix}}^2 = 
        \abs{\begin{pmatrix}
            \frac{d}{2} \\
            A \cos \theta \\
            A \sin \theta \\
            0 \\
            0 
            \end{pmatrix}}^2$$
            $$= \paren{\frac{d}{2}}^2 + A^2\cos^2 \theta + A^2 \sin^2 \theta + 0^2 + \cdots + 0^2$$
            $$= \paren{\frac{d}{2}}^2 + A^2(1) + 0
            \\ = \paren{\frac{d}{2}}^2 + A^2
            \\ = r^2$$
            $$\abs{\hat z - \hat x} = r$$

    The calculation is similar for $\abs{\hat z - \hat y}$; the only difference is the sign of $\frac{d}{2}$, and it's squared after so it doesn't matter.
    \end{proof}



    \item[(b)] If $2r = d$, there is exactly one such $\hat z$.

    \begin{proof}

    We assume $\hat x$ and $\hat y$ are placed on the $x$-axis as before.

    Suppose the point $\hat p = (p_1, p_2, \dots, p_n)$ satisfies the conditions. Then $\abs{x - p} = \abs{y - p} = r$, or $\abs{x - p}^2 = \abs{y - p}^2 = r^2$.

    Using the generalized formula for the norm, we then have $(p_1 + \frac{d}{2})^2 + p_2^2 + \cdots + p_k^2 = r^2$ and $(p_1 - \frac{d}{2})^2 + p_2^2 + \cdots + p_k^2 = r^2$, and subtracting these two equations gives us that 
    \begin{align*}
        \paren{p_1 + \frac{d}{2}^2} - \paren{p_1 - \frac{d}{2}^2} &= 0 \\
        p_1^2 + 2p_1 \frac{d}{2} + \paren{\frac{d}{2}}^2 - p_1^2 + 2 p_1 \frac{d}{2} - \paren{\frac{d}{2}}^2 &= 0 \\
        4p_1 \frac{d}{2} &= 0 \\
        p_1 d = 0
    \end{align*}

    and as $d > 0$, we find $p_1 = 0$. 

    Hence $\hat p = \paren{0, p_2, \dots, p_k}$. Now we return to our assumption, that $\abs{\hat x - \hat p}^2 = r^2$.

    \begin{align*}
        \abs{\hat x - \hat p}^2 &= r^2 \\
        \abs{\begin{pmatrix}
            -\frac{d}{2} \\
            0 \\
            \vdots \\
            0
        \end{pmatrix} - 
        \begin{pmatrix}
            0 \\
            p_2 \\
            \vdots \\
            p_k
        \end{pmatrix}}^2 &= r^2 \\
        \paren{-\frac{d}{2}}^2 + p_2^2 + \dots + p_k^2 &= r^2 = \paren{\frac{d}{2}}^2 \\
        p_2^2 + \dots + p_k^2 &= 0
    \end{align*}

    but all $p_i^2 \ge 0$, so their sum can only be $0$ if all $p_i^2 = 0$, and thus all $p_i = 0$.

    Again, the calculation is similar for $\hat y$; the only difference is the sign of $\frac{d}{2}$, and it is squared afterward.

    Therefore $\hat p = \paren{0, \dots, 0}$ satisfies $\abs{\hat z - \hat x} = \abs{\hat z - \hat y} = r$; and it is the only point which could do so.



        
    \end{proof}



    \item[(c)] If $2r < d$, there is no such $\hat z$.

    \begin{proof}
        Suppose for a contradiction that such a $\hat z$ did exist. Then we have, by hypothesis, $\abs{\hat x - \hat y} = d > 2r = \abs{\hat x - \hat z} + \abs{\hat z - \hat y} = 2r$. But the Triangle Inequality tells us that $\abs{\hat x - \hat y} \le \abs{\hat x - \hat z} + \abs{\hat z - \hat y}$, and we have a contradiction.
    \end{proof}

    \item[(d)] How must these statements be modified if $k$ is $2$ or $1$?

    \begin{solution}
        If $k = 2$, then there are two such $\hat z$ when the circles overlap; the solutions for (b) and (c) remain the same.

        If $k = 1$, then when the "spheres" overlap there are no such solutions. When $2r = d$, there is again one solution, at $\frac{x + y}{2}$. And the solution for (c) is the same.
    \end{solution}
\end{enumerate}

\end{document}